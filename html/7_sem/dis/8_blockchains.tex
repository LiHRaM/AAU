\documentclass[twocolumn]{article}

\title{Topic 13: Distributed Systems \\ \Large{Blockchains}}
\author{Hilmar G\'ustafsson}
\begin{document}
\maketitle

\section{Cryptographical Tools}
A summary of the tools necessary for Blockchains.

\subsection{Hash}
A hash function $H$ takes binary input of arbitrary length, and creates fixed-length output of it.
\begin{center}
 $H: X = \{0,1\}^* \rightarrow \{0,1\}^L$ \\ typically where $L \in \{128, 160, 256, 512\}$
\end{center}

For security purposes, it is important that a small change in the input results in a large change in the output.
Collisions exist, but they are hard to find with this property.

\subsection{Signatures using Asymmetric Cryptography}
Digital signatures are typically done using three algorithms, and a known hash $H$.

\paragraph{KeyGen} An algorithm which returns two keys: a private signing key, and a public verification key.

\paragraph{Sign} An algorithm which computes the signature of some input, given the private key.
This input is typically a hash of some data to be sent.

\paragraph{Verify} Decrypts the signature using the public key and compare the result with the hash of the received data.

\subsection{Merkle Trees}
A complete binary tree of hashes.
It is built starting from an initial set of symbols.
Also known as a hash tree.
Explots a hash function H (SHA1, MD5).
Leaves are H applied to the initial symbols.
Internal nodes are H applied to the sons of a node.
\section{Blockchain Basics}
A blockchain is a digitized, decentralized, public ledger of all cryptocurrency transactions. Constantly growing as 'mined' blocks (the most recent transactions) are recorded and added to it in chronological order, it allows market participants to keep track of digital currency transactions without central recordkeeping \footnote{Merkle tree}.
The Blockchain in essence is a collection of Blocks.
Each Block is composed of the following elements: data, a hash pointer, and a timestamp.
\subsection{Hash Pointers}
Pointer to where some info is stored.
Cryptographic hash of the info.
If we have a hash pointer, we can: (1) ask to get the info back, and (2) verify that it hasn't changed.
Similar to how programs to download are often accompanied by a hash, to verify that you downloaded the correct file.
\subsection{Consensus}
The Blockchain network is in principle completely asynchronous and decentralized.
In terms of currency, the problem to solve is double-spending, i.e. being able to spend the same money more than once.

\paragraph{Nakamoto Consensus} Assuming that 1. most nodes are honest, and 2. adding a block is computationally expensive, we can ignore received blockchains which are shorter than ours, for two reasons: 1. it was created later than mine, and 2. it can be out of sync.
This assumes that the honest nodes have more CPU power than attacker nodes.


\subsection{Proof of Work}
Provide a computational puzzle that is hard to solve when you want to add a valid block, but easy to solve when you want to verify that a block is valid.
Specifically, the new block must be a combination of a nonce and a header.
The nonce is accepted when $H(\mathtt{header})$ starts with $n$ zeros.
$n$ defines how hard it is to add a new block.
\paragraph{Example} Say the hashing algorithm is \texttt{SHA256}, and $n = 36$. Each nonce has the probability $2^{-36}$ of success. The nonce must therefore be brute-forced, but only requires one hash to verify.

\section{Blockchain Details}
\subsection{Transactions}
A transaction contains the following data:
\begin{itemize}
 \item A list of input transactions
 \item A list of tuples of the recipient public key, and the amount to send.
 \item Personal signature, signed with private key
\end{itemize}

\noindent
In order to verify a transaction, one must:

\begin{enumerate}
 \item Verify the signature using the public key of the sender
 \item Verify the signatures of each of the input transactions
 \item Ensure that the money has not been spent between the input transactions and the new transaction.
\end{enumerate}

Since the transaction is signed using the private key, only the owner of the identity can transfer the money from that point.
A single user can have an arbitrary amount of identities.

\subsection{User Miners}
A miner does the following:
\begin{enumerate}
 \item Verify all the transactions by looking that input transactions are covered and properly signed
 \item Compute the Merkle root hash for the transactions
 \item Solve the puzzle on the previous block, for immutability
 \item Broadcast the new header
 \item Go on collecting new transactions for next block
\end{enumerate}

Miners receive compensation for computing the next block. However, it is estimated that we waste \$15 million / day on energy to power the miners.

\subsection{Smart Contracts}
\paragraph{Definition} Computer protocols that facilitate, verify or enforce the negotiation or performance of a contract, or that make a contractual clause unnecessary. Define the rules and penalties around an agreement in the same way that a traditional contract does, but also automatically enforce those obligations, i.e. code is law.

\paragraph{Ethereum} Contracts are the main building blocks of Ethereum, the second most popular blockchain.
The contract is a program which lives inside the distributed Ethereum network and has its own ether balance of Ether\footnote{Ether is the cryptocurrency / cryptofuel of Ethereum}, memory and code.
Every time you send a transaction to a contract, it executes its code, which can store data, send transactions and interact with other contracts.
In order to run the contract, you create a transaction of Ether to the contract, optionally with some input information.
The contract runs until it completes or runs out of Ether.
The Ether paid is awarded to the winning miner.
Each miner runs the smart contract, and produces the same output.
The winning miner will publish the block to the rest of the network.
Other miners validate that they get the same result.
The contracts may be compiled online using the Solidity Compiler\footnote{https://remix.ethereum.org/}.

\section{Summary}
When to use blockchains:
\begin{enumerate}
 \item Are there multiple parties in the ecosystem?
 \item Is establishing trust between all the parties an issue?
 \item Is it critical to have a tamper-proof and permanent record of transactions?
 \item Are we securing the ownership or management of a finite resource?
 \item Does this ecosystem benefit from improved transparency?
\end{enumerate}

\end{document}
